\PassOptionsToPackage{unicode=true}{hyperref} % options for packages loaded elsewhere
\PassOptionsToPackage{hyphens}{url}
%
\documentclass[
]{article}
\usepackage{lmodern}
\usepackage{amssymb,amsmath}
\usepackage{ifxetex,ifluatex}
\ifnum 0\ifxetex 1\fi\ifluatex 1\fi=0 % if pdftex
  \usepackage[T1]{fontenc}
  \usepackage[utf8]{inputenc}
  \usepackage{textcomp} % provides euro and other symbols
\else % if luatex or xelatex
  \usepackage{unicode-math}
  \defaultfontfeatures{Scale=MatchLowercase}
  \defaultfontfeatures[\rmfamily]{Ligatures=TeX,Scale=1}
\fi
% use upquote if available, for straight quotes in verbatim environments
\IfFileExists{upquote.sty}{\usepackage{upquote}}{}
\IfFileExists{microtype.sty}{% use microtype if available
  \usepackage[]{microtype}
  \UseMicrotypeSet[protrusion]{basicmath} % disable protrusion for tt fonts
}{}
\makeatletter
\@ifundefined{KOMAClassName}{% if non-KOMA class
  \IfFileExists{parskip.sty}{%
    \usepackage{parskip}
  }{% else
    \setlength{\parindent}{0pt}
    \setlength{\parskip}{6pt plus 2pt minus 1pt}}
}{% if KOMA class
  \KOMAoptions{parskip=half}}
\makeatother
\usepackage{xcolor}
\IfFileExists{xurl.sty}{\usepackage{xurl}}{} % add URL line breaks if available
\IfFileExists{bookmark.sty}{\usepackage{bookmark}}{\usepackage{hyperref}}
\hypersetup{
  pdftitle={wetlandP documentation},
  pdfauthor={Adrian Wiegman},
  pdfborder={0 0 0},
  breaklinks=true}
\urlstyle{same}  % don't use monospace font for urls
\usepackage[margin=1in]{geometry}
\usepackage{graphicx,grffile}
\makeatletter
\def\maxwidth{\ifdim\Gin@nat@width>\linewidth\linewidth\else\Gin@nat@width\fi}
\def\maxheight{\ifdim\Gin@nat@height>\textheight\textheight\else\Gin@nat@height\fi}
\makeatother
% Scale images if necessary, so that they will not overflow the page
% margins by default, and it is still possible to overwrite the defaults
% using explicit options in \includegraphics[width, height, ...]{}
\setkeys{Gin}{width=\maxwidth,height=\maxheight,keepaspectratio}
\setlength{\emergencystretch}{3em}  % prevent overfull lines
\providecommand{\tightlist}{%
  \setlength{\itemsep}{0pt}\setlength{\parskip}{0pt}}
\setcounter{secnumdepth}{-2}
% Redefines (sub)paragraphs to behave more like sections
\ifx\paragraph\undefined\else
  \let\oldparagraph\paragraph
  \renewcommand{\paragraph}[1]{\oldparagraph{#1}\mbox{}}
\fi
\ifx\subparagraph\undefined\else
  \let\oldsubparagraph\subparagraph
  \renewcommand{\subparagraph}[1]{\oldsubparagraph{#1}\mbox{}}
\fi

% set default figure placement to htbp
\makeatletter
\def\fps@figure{htbp}
\makeatother


\title{wetlandP documentation}
\author{Adrian Wiegman}
\date{5/17/2020}

\begin{document}
\maketitle

This file contains a summary tables documenting the major components of
the wetlandP model. \emph{The current model version is 1.1}.

\hypertarget{state-variables}{%
\subsection{State variables}\label{state-variables}}

\begin{verbatim}
## # A tibble: 14 x 3
##    name    unit  description                 
##    <chr>   <chr> <chr>                       
##  1 IM_a    g d.w inorganic matter aboveground
##  2 IM_b    g d.w inorganic matter belowground
##  3 shootP  g P   aboveground live shoot P    
##  4 rootP   g P   belowground live root P     
##  5 litterP g P   aboveground litter P        
##  6 ROP_a   g P   refractory OP aboveground   
##  7 LOP_a   g P   labile OP aboveground       
##  8 PIP_a   g P   particulate IP aboveground  
##  9 DIP_a   g P   dissolved IP aboveground    
## 10 ROP_b   g P   refractory OP belowground   
## 11 LOP_b   g P   labile OP belowground       
## 12 PIP_b   g P   particulate IP belowground  
## 13 DIP_b   g P   dissolved IP belowground    
## 14 <NA>    <NA>  <NA>
\end{verbatim}

\hypertarget{processes}{%
\subsection{Processes}\label{processes}}

\hypertarget{flows}{%
\subsubsection{Flows}\label{flows}}

\begin{verbatim}
##                  Name                                 Value     Unit
## 1        assim_shootP                    r_assim*(k_f_sh_G)    g P/d
## 2         assim_rootP                  r_assim*(1-k_f_sh_G)    g P/d
## 3 mort_shootP2litterP                   r_mort_shoot*shootP    g P/d
## 4      mort_rootP2LOP   r_mort_root*rootP*(k_f_labile_root)    g P/d
## 5      mort_rootP2ROP r_mort_root*rootP*(1-k_f_labile_root)    g P/d
## 6              sed_IM                         r_sed_IM*IM_a g d.w./d
##                         Description Assumptions
## 1           assimilation of shoot P            
## 2                  growth of root P            
## 3                  growth of root P            
## 4       mortality of shoot P to LOP            
## 5              mortatlity of root P            
## 6 sedimentation of inorganic matter
\end{verbatim}

\hypertarget{rates}{%
\subsubsection{Rates}\label{rates}}

\begin{verbatim}
## # A tibble: 10 x 4
##    name     unit  description                    assumptions                    
##    <chr>    <chr> <chr>                          <chr>                          
##  1 r_assim  g P/d amount of DIP_b P (g) assimil~ affected by nutrient availabil~
##  2 r_decay~ 1/d   proportional rate of litter d~ affected by temperature (Wang ~
##  3 r_decay~ 1/d   proportional rate of ROP deco~ affected by temperature (Wang ~
##  4 r_decay~ 1/d   proportional rate of LOP deco~ affected by temperature (Wang ~
##  5 r_mort_~ 1/d   proportional rate of shoot de~ increases as a step function w~
##  6 r_mort_~ 1/d   proportional rate of root dea~ constant (Morris et al. 2002)  
##  7 r_adsorp g P/d ammount of DIP adsorbed to PIP affected by temperature and so~
##  8 r_diffus g P/d amount of diffusion of DIP_b ~ affected by temperature, visco~
##  9 r_sed_IM 1/d   proportional rate of sediment~ affected by settling velocity ~
## 10 r_sed_OM 1/d   proportional rate of sediment~ affected by settling velocity ~
\end{verbatim}

\hypertarget{parameters}{%
\subsection{Parameters}\label{parameters}}

\begin{verbatim}
##       Name Value  Unit                      Description
## 1   # name value units                      description
## 2   TW_max    23  degC max daily mean water tempurature
## 3   TW_min     4  degC min daily mean water tempurature
## 4 TA_h_max    28  degC   max daily high air tempurature
## 5 TA_h_min    -2  degC   min daily high air tempurature
## 6 TA_l_max    17  degC    max daily low air tempurature
##                                                         Assumptions
## 1                                                       assumptions
## 2        fit to site data or assume equal to mean air temp on Aug 1
## 3              fit to site data or assume equal to 4 degC on Jan 31
## 4 fit to data from local city/town (e.g. https://weatherspark.com/)
## 5                                                                ""
## 6                                                                ""
\end{verbatim}

\hypertarget{differential-equations}{%
\subsection{Differential Equations}\label{differential-equations}}

Differential Equations for the model are generated from stoicheometry
matrix of the \textbf{state variables} and \textbf{process flows} (see
\textbf{``mass balance''}).

\hypertarget{mass-balance}{%
\subsection{Mass balance}\label{mass-balance}}

Differential Equations for the model are generated from stoicheometry
matrix of the \textbf{state variables} (state or states for short) and
\textbf{process flows} (see \texttt{stoicheometry.xlsx}). In this matrix
the modeler enters a value of 1 (adding to a state), -1 (subtracting
from state) or blank (not interacting with a state) for each combination
of a state variable and a process flow. The table below contains the
stoicheometry matrix for the current model. Note the column
\texttt{balance} is the row sum for a given process, values above or
below than zero indicates that a process adds/removes mass from the
model domain, while a balance of zero indicates that a process is
conservative (does not affect the total mass in the domain).

\begin{verbatim}
## # A tibble: 27 x 15
##    balance `variables (rig~  IM_a  IM_b shootP rootP litterP ROP_a LOP_a PIP_a
##      <dbl> <chr>            <dbl> <dbl>  <dbl> <dbl>   <dbl> <dbl> <dbl> <dbl>
##  1       0 assim_shootP        NA    NA      1    NA      NA    NA    NA    NA
##  2       0 assim_rootP         NA    NA     NA     1      NA    NA    NA    NA
##  3       0 mort_shootP2lit~    NA    NA     -1    NA       1    NA    NA    NA
##  4       0 mort_rootP2LOP      NA    NA     NA    -1      NA    NA    NA    NA
##  5       0 mort_rootP2ROP      NA    NA     NA    -1      NA    NA    NA    NA
##  6       0 sed_IM              -1     1     NA    NA      NA    NA    NA    NA
##  7       0 sed_PIP             NA    NA     NA    NA      NA    NA    NA    -1
##  8       0 sed_LOP             NA    NA     NA    NA      NA    NA    -1    NA
##  9       0 sed_ROP             NA    NA     NA    NA      NA    -1    NA    NA
## 10       0 dec_litter2LOP_a    NA    NA     NA    NA      -1    NA     1    NA
## # ... with 17 more rows, and 5 more variables: DIP_a <dbl>, ROP_b <dbl>,
## #   LOP_b <dbl>, PIP_b <dbl>, DIP_b <dbl>
\end{verbatim}

\end{document}
