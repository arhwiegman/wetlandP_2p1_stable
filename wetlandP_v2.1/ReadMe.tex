% Options for packages loaded elsewhere
\PassOptionsToPackage{unicode}{hyperref}
\PassOptionsToPackage{hyphens}{url}
%
\documentclass[
]{article}
\usepackage{lmodern}
\usepackage{amssymb,amsmath}
\usepackage{ifxetex,ifluatex}
\ifnum 0\ifxetex 1\fi\ifluatex 1\fi=0 % if pdftex
  \usepackage[T1]{fontenc}
  \usepackage[utf8]{inputenc}
  \usepackage{textcomp} % provide euro and other symbols
\else % if luatex or xetex
  \usepackage{unicode-math}
  \defaultfontfeatures{Scale=MatchLowercase}
  \defaultfontfeatures[\rmfamily]{Ligatures=TeX,Scale=1}
\fi
% Use upquote if available, for straight quotes in verbatim environments
\IfFileExists{upquote.sty}{\usepackage{upquote}}{}
\IfFileExists{microtype.sty}{% use microtype if available
  \usepackage[]{microtype}
  \UseMicrotypeSet[protrusion]{basicmath} % disable protrusion for tt fonts
}{}
\makeatletter
\@ifundefined{KOMAClassName}{% if non-KOMA class
  \IfFileExists{parskip.sty}{%
    \usepackage{parskip}
  }{% else
    \setlength{\parindent}{0pt}
    \setlength{\parskip}{6pt plus 2pt minus 1pt}}
}{% if KOMA class
  \KOMAoptions{parskip=half}}
\makeatother
\usepackage{xcolor}
\IfFileExists{xurl.sty}{\usepackage{xurl}}{} % add URL line breaks if available
\IfFileExists{bookmark.sty}{\usepackage{bookmark}}{\usepackage{hyperref}}
\hypersetup{
  pdftitle={Documentation -- wetlandP\_v2.1},
  pdfauthor={Adrian Wiegman},
  hidelinks,
  pdfcreator={LaTeX via pandoc}}
\urlstyle{same} % disable monospaced font for URLs
\usepackage[margin=1in]{geometry}
\usepackage{longtable,booktabs}
% Correct order of tables after \paragraph or \subparagraph
\usepackage{etoolbox}
\makeatletter
\patchcmd\longtable{\par}{\if@noskipsec\mbox{}\fi\par}{}{}
\makeatother
% Allow footnotes in longtable head/foot
\IfFileExists{footnotehyper.sty}{\usepackage{footnotehyper}}{\usepackage{footnote}}
\makesavenoteenv{longtable}
\usepackage{graphicx}
\makeatletter
\def\maxwidth{\ifdim\Gin@nat@width>\linewidth\linewidth\else\Gin@nat@width\fi}
\def\maxheight{\ifdim\Gin@nat@height>\textheight\textheight\else\Gin@nat@height\fi}
\makeatother
% Scale images if necessary, so that they will not overflow the page
% margins by default, and it is still possible to overwrite the defaults
% using explicit options in \includegraphics[width, height, ...]{}
\setkeys{Gin}{width=\maxwidth,height=\maxheight,keepaspectratio}
% Set default figure placement to htbp
\makeatletter
\def\fps@figure{htbp}
\makeatother
\setlength{\emergencystretch}{3em} % prevent overfull lines
\providecommand{\tightlist}{%
  \setlength{\itemsep}{0pt}\setlength{\parskip}{0pt}}
\setcounter{secnumdepth}{-\maxdimen} % remove section numbering
\ifluatex
  \usepackage{selnolig}  % disable illegal ligatures
\fi

\title{Documentation -- wetlandP\_v2.1}
\author{Adrian Wiegman}
\date{30 October, 2021}

\begin{document}
\maketitle

{
\setcounter{tocdepth}{2}
\tableofcontents
}
\textbf{Note this project is still under development}

\hypertarget{about}{%
\subsection{About}\label{about}}

The wetlandP\_v2.1 model is an ordinary differential equation model
developed for decadal phosphorus (P) retention simulations in riparian
wetlands with a range of soil and hydrologic conditions.

\hypertarget{funding}{%
\subsubsection{Funding}\label{funding}}

\textbf{Quantifying phosphorus retention in restored riparian wetlands
of the Lake Champlain Basin}

EPA Grant: LC00A00394

Job Cost Code: 995-002-001

\hypertarget{project-team}{%
\paragraph{Project Team:}\label{project-team}}

Eric D. Roy, Ph.D., University of Vermont (Principle Investigator)

Breck Bowden, Ph.D., University of Vermont (Investigator)

Kristin Underwood, Ph.D., University of Vermont (Investigator)

Adrian Wiegman, M.S., University of Vermont (PhD Candidate)

\hypertarget{granting-agency}{%
\paragraph{Granting Agency:}\label{granting-agency}}

Lake Champlain Basin Program

54 West Shore Road

Grand Isle, VT 05458

\hypertarget{buildnotes}{%
\subsubsection{Buildnotes}\label{buildnotes}}

Model version: wetlandP\_v2.1

The data for this project are hosted at the authors private github
repository: \url{https://github.com/arhwiegman/wetlandP}

The current version of the model is available within the wetlandP
repository:
\url{https://github.com/arhwiegman/wetlandP/tree/master/model_versions/wetlandP_v2/wetlandP_v2.1}

This repository will be made public upon publication of this work.

\hypertarget{status-of-this-version}{%
\paragraph{Status (of this version)}\label{status-of-this-version}}

\begin{enumerate}
\def\labelenumi{\arabic{enumi}.}
\tightlist
\item
  Switches have been added to toggle process flow rates see
  \texttt{IO\_} in parameters.
\item
  All hydroclimatic variabels are read in as input data then fit to an
  \texttt{approxfun} so that values can be interpolated at any discrete
  time point. See \texttt{scripts/preprocssing} for preparation of
  hydroclimatic input table.\\
\item
  Moved away from the use of langmuir model of adsorption, instead
  DIP\_E can either be entered as a constant, or calculated using a
  power model fit to final intact core SRP and (Ex\_max - Ex)/(PSR).
\item
  New script added to keep track of and, when needed, install
  dependancies.
\item
  \texttt{Packrat} is no longer being used \texttt{pacman} is being used
  to install load packages.
\item
  Revised biomass growth equations to include temperature effects on
  growth rate and mortality, and omit water level and self crowding
  effects on growth rate.
\end{enumerate}

\hypertarget{potential-next-steps-for-future-versions}{%
\paragraph{Potential Next steps (for future
versions)}\label{potential-next-steps-for-future-versions}}

\begin{enumerate}
\def\labelenumi{\arabic{enumi}.}
\tightlist
\item
  Incorporate subroutine that takes raw climate data from weather
  stations and water level data and prepares a proper input table.
\item
  Add subroutines for P flows due to periphyton, bioturbation
\item
  Add subroutine to toggle aerobic/anearobic sediments and associated
  changes in DIP\_E
\item
  Improve computational efficiency (decrease simulation time).
\item
  Add option to use NRCS Soil Survey Data and/or Farming Frequency
  and/or Years since farming to initialize state variables.
\item
  Add the ability to take hydrologic parameters such as inundation
  frequency and depth, and produce a synthetic flood hydrograph.
\item
  Add subroutines for management including: fertilizer additon, biomass
  harvest, ditch plugging, and berm removal
\item
  Implement the wetlandP\_v2.1 R project with \texttt{packrat} to avoid
  compatability issues among local R package versions (Ushey et
  al.~2018).
\item
  Add subroutines for freeze/thaw
\end{enumerate}

\hypertarget{getting-started}{%
\subsection{Getting Started}\label{getting-started}}

The wetlandP\_v2.1 model is written in R version 4.0.1 (2020-06-06) --
``See Things Now'' using Rstudio (v. 1.2.1). An R project file (.Rpoj)
is the user interface.To run the model, click on the wetlandP.Rproj
file. This will open up Rstudio with the wetlandP\_v2.1 working
directory.

\hypertarget{running-the-model}{%
\subsubsection{Running the Model}\label{running-the-model}}

Read the remainder of this section for details on how to edit parameters
and run the model and view model outputs.

\hypertarget{file-structure}{%
\subsubsection{File Structure}\label{file-structure}}

A current list of the working directory is given below.

\begin{verbatim}
##  [1] "documentation"                     "Documentation – wetlandP_v2.1.pdf"
##  [3] "inputs"                            "outputs"                          
##  [5] "ReadMe.docx"                       "ReadMe.html"                      
##  [7] "ReadMe.Rmd"                        "ReadMe.tex"                       
##  [9] "Rplot.pdf"                         "Rplot01.pdf"                      
## [11] "Rplot02.pdf"                       "scripts"                          
## [13] "scripts - 2021-10-25"              "wetlandP.Rproj"
\end{verbatim}

The sections below describe the \texttt{documentation},
\texttt{scripts}, \texttt{inputs}, and \texttt{outputs} in more detail.

\hypertarget{dependacies}{%
\subsubsection{Dependacies}\label{dependacies}}

A script called \texttt{dependancies.R} uses \texttt{pacman} to check
for and install required R packages. The wetlandP\_v2.1 simulations are
implemented with the \texttt{deSolve} package (Soetaert et al.~2010),
data management and plotting is implemented with the \texttt{tidyverse}
packages (Wickham et al.~2019). For more details on the packages used
see the \texttt{scripts/dependancies.R} file.

Upon running this file you will see the following console output

\begin{verbatim}
## successfully loaded dependant R packages:
##  [1] "ggrepel"            "ecolMod"            "diagram"           
##  [4] "shape"              "rootSolve"          "rlang"             
##  [7] "Evapotranspiration" "soiltexture"        "zoo"               
## [10] "diffeqr"            "deSolve"            "pacman"            
## [13] "forcats"            "stringr"            "dplyr"             
## [16] "purrr"              "readr"              "tidyr"             
## [19] "tibble"             "ggplot2"            "tidyverse"
\end{verbatim}

The \texttt{dependancies.R} script also creates an object called
depedancy\_citations. You can \texttt{print} this and copy it to add the
package citations to a bibliography.

\hypertarget{documentation-folder}{%
\subsubsection{Documentation Folder}\label{documentation-folder}}

Currently the most detailed documentation of the model is within the
model scripts. However this folder contains tables that document model
variables, values, assumptions, etc\ldots{} for major varaible types in
the model.

\begin{verbatim}
##  [1] "fig1_states_W0_B0_G0.png"             
##  [2] "fig2_states_W0_B0_G1.png"             
##  [3] "fig3_states_W0_B1_G0.png"             
##  [4] "fig4_states_W0_B1_G1.png"             
##  [5] "fig5_hydroclimate_static_W1_B0_G0.png"
##  [6] "fig6_hydroclimate_W1_B0_G0.png"       
##  [7] "fig7_states_W1_B1_G1.png"             
##  [8] "fig8_DIP_A_W1_B1_G1.png"              
##  [9] "function_tests.xlsx"                  
## [10] "generate_documentation_tables.R"      
## [11] "parameters.csv"                       
## [12] "parameters.md"                        
## [13] "parameters_local.csv"                 
## [14] "parameters_local.md"                  
## [15] "parameters_local.R"                   
## [16] "parameters_simulation.csv"            
## [17] "parameters_simulation.md"             
## [18] "parameters_simulation.R"              
## [19] "parameters_stochastic.csv"            
## [20] "parameters_stochastic.md"             
## [21] "parameters_stochastic.R"              
## [22] "parameters_table.R"                   
## [23] "parameters_universal_constant.csv"    
## [24] "parameters_universal_constant.md"     
## [25] "parameters_universal_constants.R"     
## [26] "processes.csv"                        
## [27] "processes.md"                         
## [28] "stochastic.csv"                       
## [29] "stochastic.md"                        
## [30] "stoicheometry.xlsx"                   
## [31] "stoicheometry_complex.xlsx"           
## [32] "superceded"                           
## [33] "wetlandP_v2.1_Conceptual_Diagram.png"
\end{verbatim}

\hypertarget{scripts-folder}{%
\subsubsection{Scripts Folder}\label{scripts-folder}}

\begin{verbatim}
##  [1] "_implementations" "_postprocessing"  "_preprocessing"   "_sourcecode"     
##  [5] "dependancies.R"   "fns"              "functions.R"      "initialize.R"    
##  [9] "model.R"          "parameters.R"     "subroutines.R"    "xecute.R"
\end{verbatim}

In addition to the files above the scripts folder holds four folders.
\texttt{\_implementations} contains high level scripts used to call the
model and edit inputs and outputs. \texttt{\_preprocessing} contains
scripts to conduct statistical analysis to general parameter estimates
or to calculate the hydroclimate forcing tables.
\texttt{\_postprocessing} contains scripts to analyze model outputs.
\texttt{\_sourcecodes} contains copies of the model source codes in the
main \texttt{scripts} folder. These are kept in case the user makes
manual edits to the source codes that cause errors.

The table below provides a description of each of the source codes used
by the model.

\begin{longtable}[]{@{}ll@{}}
\toprule
\begin{minipage}[b]{0.47\columnwidth}\raggedright
name\strut
\end{minipage} & \begin{minipage}[b]{0.47\columnwidth}\raggedright
description\strut
\end{minipage}\tabularnewline
\midrule
\endhead
\begin{minipage}[t]{0.47\columnwidth}\raggedright
\texttt{xecute.R}\strut
\end{minipage} & \begin{minipage}[t]{0.47\columnwidth}\raggedright
High level script to load source code, execute simulation and manage
data outputs. This script must be run to implement the model. To run the
file: in \texttt{Rstudio} with the \texttt{xecute.R} file open press
\texttt{cmd/crtl\ +\ shift\ +\ enter}\strut
\end{minipage}\tabularnewline
\begin{minipage}[t]{0.47\columnwidth}\raggedright
\texttt{parameters.R}\strut
\end{minipage} & \begin{minipage}[t]{0.47\columnwidth}\raggedright
The main way to manipulate outputs. This includes both nurmerical
constants to be used in model calculations as well as model run
specifications (e.g.~static or dynamic, simulation time) see
\texttt{fn\_edit\_parameter\_values} to change individual parameters for
before a given run.\strut
\end{minipage}\tabularnewline
\begin{minipage}[t]{0.47\columnwidth}\raggedright
\texttt{model.R}\strut
\end{minipage} & \begin{minipage}[t]{0.47\columnwidth}\raggedright
Contains the high level functions that controling flow of subroutines in
the wetlandP\_v2.1 model. See subroutines for details of model
calculations.\strut
\end{minipage}\tabularnewline
\begin{minipage}[t]{0.47\columnwidth}\raggedright
\texttt{initialize.R}\strut
\end{minipage} & \begin{minipage}[t]{0.47\columnwidth}\raggedright
initializes the model state variables based on the parameter values and
functions provided.\strut
\end{minipage}\tabularnewline
\begin{minipage}[t]{0.47\columnwidth}\raggedright
\texttt{subroutines.R}\strut
\end{minipage} & \begin{minipage}[t]{0.47\columnwidth}\raggedright
a series of subroutines that calculate new values of variables in the
model based on functions, parameters and variable values in the model
environment.\strut
\end{minipage}\tabularnewline
\begin{minipage}[t]{0.47\columnwidth}\raggedright
\texttt{functions.R}\strut
\end{minipage} & \begin{minipage}[t]{0.47\columnwidth}\raggedright
a high level script that sources other functions.\strut
\end{minipage}\tabularnewline
\begin{minipage}[t]{0.47\columnwidth}\raggedright
\texttt{functions/fns\_X.R}\strut
\end{minipage} & \begin{minipage}[t]{0.47\columnwidth}\raggedright
functions pertaining to \texttt{X} aspect of the model (such as
``processes'')\strut
\end{minipage}\tabularnewline
\begin{minipage}[t]{0.47\columnwidth}\raggedright
\texttt{dependancies.R}\strut
\end{minipage} & \begin{minipage}[t]{0.47\columnwidth}\raggedright
checks for and installs required R packages\strut
\end{minipage}\tabularnewline
\bottomrule
\end{longtable}

\hypertarget{inputs-folder}{%
\subsubsection{Inputs Folder}\label{inputs-folder}}

Inputs are taken as \texttt{.csv} (comma separated values) files and
read in using the function \texttt{readr::read\_csv()}. There are two
kinds of input file, \texttt{df.hydroclimate...csv} and
\texttt{df.parameters...csv}.\texttt{hydroclimate} files provide a time
series of forcing data with the top row as the variable name and each
column containing the values for the variable at time \texttt{t}, at
least one column must be named \texttt{t}. \texttt{parameters} files are
tables with the columns from left to right variable name, default value,
units, description, assumptioms, random distribution function name and
inputs.

\begin{verbatim}
##  [1] "CH3_simulation_vars.Rmd"                          
##  [2] "coreflux"                                         
##  [3] "df.climate.subdaily.csv"                          
##  [4] "df.hydroclimate.1m.LC.0.csv"                      
##  [5] "df.hydroclimate.1m.LC.0.Rdata"                    
##  [6] "df.hydroclimate.1m.LC.0x1p2.csv"                  
##  [7] "df.hydroclimate.1m.LC.0x1p2.Rdata"                
##  [8] "df.hydroclimate.1m.LC.1.csv"                      
##  [9] "df.hydroclimate.1m.LC.1.Rdata"                    
## [10] "df.hydroclimate.1m.LC.1x1p2.csv"                  
## [11] "df.hydroclimate.1m.LC.1x1p2.Rdata"                
## [12] "df.hydroclimate.1m.LC.2.csv"                      
## [13] "df.hydroclimate.1m.LC.2.Rdata"                    
## [14] "df.hydroclimate.1m.LC.2x1p2.csv"                  
## [15] "df.hydroclimate.1m.LC.2x1p2.Rdata"                
## [16] "df.hydroclimate.1m.LC.3.csv"                      
## [17] "df.hydroclimate.1m.LC.3.Rdata"                    
## [18] "df.hydroclimate.1m.LC.3x1p2.csv"                  
## [19] "df.hydroclimate.1m.LC.3x1p2.Rdata"                
## [20] "df.hydroclimate.1m.LC.4.csv"                      
## [21] "df.hydroclimate.1m.LC.4.Rdata"                    
## [22] "df.hydroclimate.1m.LC.4x1p2.csv"                  
## [23] "df.hydroclimate.1m.LC.4x1p2.Rdata"                
## [24] "df.hydroclimate.1m.OCD.0.csv"                     
## [25] "df.hydroclimate.1m.OCD.0.Rdata"                   
## [26] "df.hydroclimate.1m.OCD.0x1p2.csv"                 
## [27] "df.hydroclimate.1m.OCD.0x1p2.Rdata"               
## [28] "df.hydroclimate.1m.OCD.1.csv"                     
## [29] "df.hydroclimate.1m.OCD.1.Rdata"                   
## [30] "df.hydroclimate.1m.OCD.1x1p2.csv"                 
## [31] "df.hydroclimate.1m.OCD.1x1p2.Rdata"               
## [32] "df.hydroclimate.1m.OCD.2.csv"                     
## [33] "df.hydroclimate.1m.OCD.2.Rdata"                   
## [34] "df.hydroclimate.1m.OCD.2x1p2.csv"                 
## [35] "df.hydroclimate.1m.OCD.2x1p2.Rdata"               
## [36] "df.hydroclimate.1m.OCD.3.csv"                     
## [37] "df.hydroclimate.1m.OCD.3.Rdata"                   
## [38] "df.hydroclimate.1m.OCD.3x1p2.csv"                 
## [39] "df.hydroclimate.1m.OCD.3x1p2.Rdata"               
## [40] "df.hydroclimate.1m.OCD.4.csv"                     
## [41] "df.hydroclimate.1m.OCD.4.Rdata"                   
## [42] "df.hydroclimate.1m.OCD.4x1p2.csv"                 
## [43] "df.hydroclimate.1m.OCD.4x1p2.Rdata"               
## [44] "df.hydroclimate.1m.OCSP.0.csv"                    
## [45] "df.hydroclimate.1m.OCSP.0.Rdata"                  
## [46] "df.hydroclimate.1m.OCSP.0x1p2.csv"                
## [47] "df.hydroclimate.1m.OCSP.0x1p2.Rdata"              
## [48] "df.hydroclimate.1m.OCSP.1.csv"                    
## [49] "df.hydroclimate.1m.OCSP.1.Rdata"                  
## [50] "df.hydroclimate.1m.OCSP.1x1p2.csv"                
## [51] "df.hydroclimate.1m.OCSP.1x1p2.Rdata"              
## [52] "df.hydroclimate.1m.OCSP.2.csv"                    
## [53] "df.hydroclimate.1m.OCSP.2.Rdata"                  
## [54] "df.hydroclimate.1m.OCSP.2x1p2.csv"                
## [55] "df.hydroclimate.1m.OCSP.2x1p2.Rdata"              
## [56] "df.hydroclimate.1m.OCSP.3.csv"                    
## [57] "df.hydroclimate.1m.OCSP.3.Rdata"                  
## [58] "df.hydroclimate.1m.OCSP.3x1p2.csv"                
## [59] "df.hydroclimate.1m.OCSP.3x1p2.Rdata"              
## [60] "df.hydroclimate.1m.OCSP.4.csv"                    
## [61] "df.hydroclimate.1m.OCSP.4.Rdata"                  
## [62] "df.hydroclimate.1m.OCSP.4x1p2.csv"                
## [63] "df.hydroclimate.1m.OCSP.4x1p2.Rdata"              
## [64] "df.hydroclimate.day.LC.csv"                       
## [65] "df.hydroclimate.day.LC.Rdata"                     
## [66] "df.hydroclimate_dynamic.csv"                      
## [67] "df.hydroclimate_dynamic.csv.xlsx"                 
## [68] "df.hydroclimate_static.csv"                       
## [69] "df.hydroclimate_static.csv.xlsx"                  
## [70] "df.hydroclimate_steady_state_sensitivity.csv"     
## [71] "df.hydroclimate_steady_state_sensitivity.csv.xlsx"
## [72] "df.pred.csv"                                      
## [73] "df.stage_volume_discharge.csv"                    
## [74] "df.stage_volume_discharge.Rdata"                  
## [75] "hydrosummary.txt"                                 
## [76] "lcbp_sites"                                       
## [77] "readme.txt"                                       
## [78] "simulation_steps.txt"
\end{verbatim}

\hypertarget{outputs-folder}{%
\subsubsection{Outputs Folder}\label{outputs-folder}}

The model saves outputs with a prefix then the simulation then a
timestamp. wetlandP\_v2.1 produces three types of output:

\begin{longtable}[]{@{}lll@{}}
\toprule
\begin{minipage}[b]{0.30\columnwidth}\raggedright
prefix\strut
\end{minipage} & \begin{minipage}[b]{0.30\columnwidth}\raggedright
extension\strut
\end{minipage} & \begin{minipage}[b]{0.30\columnwidth}\raggedright
description\strut
\end{minipage}\tabularnewline
\midrule
\endhead
\begin{minipage}[t]{0.30\columnwidth}\raggedright
\texttt{sim\_}\strut
\end{minipage} & \begin{minipage}[t]{0.30\columnwidth}\raggedright
\texttt{.Rdata}\strut
\end{minipage} & \begin{minipage}[t]{0.30\columnwidth}\raggedright
an image of the R environment objects saved upon execution of the model
run. Use \texttt{load("sim\_{[}run\ name{]}.Rdata")} in R to load the
environment objects for the simulation.\strut
\end{minipage}\tabularnewline
\begin{minipage}[t]{0.30\columnwidth}\raggedright
\texttt{fig\_}\strut
\end{minipage} & \begin{minipage}[t]{0.30\columnwidth}\raggedright
\texttt{.png}\strut
\end{minipage} & \begin{minipage}[t]{0.30\columnwidth}\raggedright
time series plots of variables\strut
\end{minipage}\tabularnewline
\begin{minipage}[t]{0.30\columnwidth}\raggedright
\texttt{outputs\_}\strut
\end{minipage} & \begin{minipage}[t]{0.30\columnwidth}\raggedright
\texttt{.csv}\strut
\end{minipage} & \begin{minipage}[t]{0.30\columnwidth}\raggedright
a comma delimited data table of variable values along the time series
the model run\strut
\end{minipage}\tabularnewline
\bottomrule
\end{longtable}

A snapshot of the outputs folder is given below:

\begin{verbatim}
## [1] "df.obs_vs_simpars_lcbp_siphon (2).csv"                         
## [2] "df.obs_vs_simpars_lcbp_siphon (2)_OCSP.4.csv"                  
## [3] "df.obs_vs_simpars_lcbp_siphon (2)_OCSP.4.xlsx"                 
## [4] "df.sim.outs_steady_state_sensitivity_nsims10_2021-10-14.csv"   
## [5] "df.sim.outs_steady_state_sensitivity_nsims1000_2021-10-14.csv" 
## [6] "df.sim.outs_steady_state_sensitivity_nsims10000_2021-10-16.csv"
\end{verbatim}

\hypertarget{model-variables}{%
\subsection{Model Variables}\label{model-variables}}

This section contains summary tables defining the major variables of
wetlandP\_v2.1. A conceptual diagram of model domain, compartments,
state variables, and processes is given below. Flows of phosphorus are
represented by lines with arrows, the associated process for each flow
is labeled in italics. State variables are represented in boxes with
rounded edges. See the sections below for detailed definitions of
states, processes, etc.

\includegraphics{C:/Workspace/wetlandP/model_versions/wetlandP_v2/wetlandP_v2.1/documentation/wetlandP_v2.1_Conceptual_Diagram.png}

The figure below shows a time series plot of the model state variables.

\includegraphics{ReadMe_files/figure-latex/unnamed-chunk-9-1.pdf}

The default parameters for wetlandP\_v2.1 currently produce state
variable values close to what has been observed in the field. This
includes dissolved inorganic P concentrations (see plot below)

\includegraphics{ReadMe_files/figure-latex/unnamed-chunk-10-1.pdf}

\hypertarget{state-variables}{%
\subsubsection{State variables}\label{state-variables}}

\begin{longtable}[]{@{}lll@{}}
\caption{state variables calculated with ordinary differential
equations}\tabularnewline
\toprule
name & unit & description\tabularnewline
\midrule
\endfirsthead
\toprule
name & unit & description\tabularnewline
\midrule
\endhead
IM\_a & g d.w & inorganic matter aboveground\tabularnewline
IM\_b & g d.w & inorganic matter belowground\tabularnewline
shootP & g P & aboveground live shoot P\tabularnewline
rootP & g P & belowground live root P\tabularnewline
litterP & g P & aboveground litter P\tabularnewline
ROP\_a & g P & refractory OP aboveground\tabularnewline
LOP\_a & g P & labile OP aboveground\tabularnewline
PIP\_a & g P & particulate IP aboveground\tabularnewline
DIP\_a & g P & dissolved IP aboveground\tabularnewline
ROP\_b & g P & refractory OP belowground\tabularnewline
LOP\_b & g P & labile OP belowground\tabularnewline
PIP\_b & g P & particulate IP belowground\tabularnewline
DIP\_b & g P & dissolved IP belowground\tabularnewline
\bottomrule
\end{longtable}

\hypertarget{processes}{%
\subsubsection{Processes}\label{processes}}

\hypertarget{flows}{%
\paragraph{Flows}\label{flows}}

\begin{longtable}[]{@{}lllll@{}}
\caption{process flows (adding or subtracting from state
variables)}\tabularnewline
\toprule
Name & Value & Unit & Description & Assumptions\tabularnewline
\midrule
\endfirsthead
\toprule
Name & Value & Unit & Description & Assumptions\tabularnewline
\midrule
\endhead
Q\_in & IO\_Q\_in*Q\_in & m\^{}3/d & surface water lateral inflow & note
all hydrologic should be positive magnitude values, they are then
multiplied by 1, 0, -1 in differential equaitions\tabularnewline
Q\_out & IO\_Q\_out*Q\_out & m\^{}3/d & surface water lateral outflow
&\tabularnewline
Q\_ground & IO\_Q\_ground*Q\_ground & m\^{}3/d & net vertical flow from
groundwater (percolation - infiltration) &\tabularnewline
Q\_precip & IO\_Q\_precip*Q\_precip & m\^{}3/d & direct precipitation
&\tabularnewline
Q\_ET & IO\_Q\_ET*Q\_ET & m\^{}3/d & evapotranspiration precipitation
&\tabularnewline
assim\_shootP & IO\_assim\_shootP*r\_assim*BM*k\_BM2P*k\_f\_G\_shoot & g
P/d & assimilation of shoot P &\tabularnewline
assim\_rootP & IO\_assim\_rootP*r\_assim*BM*k\_BM2P*(1-k\_f\_G\_shoot) &
g P/d & growth of root P &\tabularnewline
mort\_shootP2litterP & IO\_mort\_shootP2litterP*r\_mort\_shoot*shootP &
g P/d & growth of root P &\tabularnewline
mort\_rootP2LOP &
IO\_mort\_rootP2LOP*r\_mort\_root*rootP*(k\_f\_labile\_root) & g P/d &
mortality of shoot P to LOP &\tabularnewline
mort\_rootP2ROP &
IO\_mort\_rootP2ROP*r\_mort\_root*rootP*(1-k\_f\_labile\_root) & g P/d &
mortatlity of root P &\tabularnewline
sed\_IM & IO\_sed\_IM*r\_sed\_IM*IM\_a & g d.w./d & sedimentation of
inorganic matter &\tabularnewline
sed\_PIP & IO\_sed\_IM*r\_sed\_IM*PIP\_a & g P/d & sedimentation of
inorganic P &\tabularnewline
sed\_LOP & IO\_sed\_OM*r\_sed\_OM*LOP\_a & g P/d & sedimentation of
labile organic P &\tabularnewline
sed\_ROP & IO\_sed\_OM*r\_sed\_OM*ROP\_a & g P/d & sedimentation of
refractory organic P &\tabularnewline
dec\_litter2LOP\_a &
IO\_decay\_litter*r\_decay\_litter*litterP*(k\_f\_labile\_litter) & &
decomposition of litter P to labile organic P &\tabularnewline
dec\_litter2ROP\_a &
IO\_decay\_litter*r\_decay\_litter*litterP*(1-k\_f\_labile\_litter) & &
decomposition of litter P to refractory organic P &\tabularnewline
dec\_LOP\_a & IO\_decay\_LOP*r\_decay\_LOP*LOP\_a & g P/d &
decomposition of labile OP to DIP &\tabularnewline
dec\_ROP\_a & IO\_decay\_ROP*r\_decay\_ROP*ROP\_a & g P/d &
decomposition of refractory OP to labile OP &\tabularnewline
dec\_LOP\_b & IO\_decay\_LOP*r\_decay\_LOP*LOP\_b & g P/d &
decomposition of labile OP to DIP &\tabularnewline
dec\_ROP\_b & IO\_decay\_ROP*r\_decay\_ROP*ROP\_b & g P/d &
decomposition of refractory OP to labile OP &\tabularnewline
diff\_DIP\_b2a & IO\_diffus*r\_diffus*DIP\_b & g P/d & diffusion of DIP
from b to a &\tabularnewline
sorp\_DIP2PIP\_b & IO\_adsorp*r\_adsorp*V\_w\_b & g P/d & adsorption of
DIP onto PIP &\tabularnewline
in\_IM & IO\_in\_IM*Q\_in*ISS & g d.w./d & inflow of inorganic matter as
ISS &\tabularnewline
in\_PIP & IO\_in\_PIP*(in\_IM*k\_ISS2P + Q\_in*SRP*k\_SRP2PIP) & g P/d &
inflow of PIP &\tabularnewline
in\_LOP & IO\_in\_LOP*Q\_in*OSS*k\_BM2P*(k\_f\_labile) & g P/d & inflow
of labile organic P &\tabularnewline
in\_ROP & IO\_in\_ROP*Q\_in*OSS*k\_BM2P*(1-k\_f\_labile) & g P/d &
inflow of recalcitrant organic P &\tabularnewline
in\_DIP & IO\_in\_DIP*Q\_in*SRP & g P/d & inflow of dissolved inorganic
P &\tabularnewline
out\_IM & IO\_out\_IM*Q\_out*IM\_a/V\_w\_a & g P/d & outflow of IM
&\tabularnewline
out\_PIP & IO\_out\_PIP*Q\_out*PIP\_a/V\_w\_a & g P/d & outflow of PIP
&\tabularnewline
out\_LOP & IO\_out\_LOP*Q\_out*LOP\_a/V\_w\_a & g P/d & outflow of LOP
&\tabularnewline
out\_ROP & IO\_out\_ROP*Q\_out*ROP\_a/V\_w\_a & g P/d & outflow of ROP
&\tabularnewline
out\_DIP & IO\_out\_DIP*Q\_out*DIP\_a/V\_w\_a & g P/d & outflow of DIP
&\tabularnewline
\bottomrule
\end{longtable}

\hypertarget{rates}{%
\paragraph{Rates}\label{rates}}

\begin{longtable}[]{@{}llll@{}}
\caption{process rates (calculated as a function of forcing variables,
intermediate variables and state variables)}\tabularnewline
\toprule
name & unit & description & assumptions\tabularnewline
\midrule
\endfirsthead
\toprule
name & unit & description & assumptions\tabularnewline
\midrule
\endhead
r\_assim & g P/d & amount of DIP\_b P (g) assimilated by macrophyte
plants as net primary productivity & always affected by temperature, DIP
availability, and optionally affected by water level, and self-crowding
or shading (Marois \& Mitsch 2016; Morris et al.~2002; Wiegman et
al.~2019)\tabularnewline
r\_decay\_litter & 1/d & proportional rate of litter decomposition &
affected by temperature (Wang et al.~2003)\tabularnewline
r\_decay\_ROP & 1/d & proportional rate of ROP decomposition & same as
r\_decay\_litter\tabularnewline
r\_decay\_LOP & 1/d & proportional rate of LOP decomposition & same as
r\_decay\_litter\tabularnewline
r\_mort\_shoot & 1/d & proportional rate of shoot death & affected by
temperature, increases as a step function when temp drops below
threshold (based on field observations) (Wiegman et
al.~2019)\tabularnewline
r\_mort\_root & 1/d & proportional rate of root death & affected by
temperature (Morris et al.~2002)\tabularnewline
r\_adsorp\_b & g P/d & ammount of DIP adsorbed to PIP belowground &
affected by temperature and equilibrium DIP (Wang et al.~2003);
equilibrium DIP can be set as a parameter, or calculated as a function
from maximum P storage capacity (Ex\_max, variable or constant), and the
currently adsorbed P (Ex = PIP\_b).\tabularnewline
r\_diffus & g P/d & amount of diffusion of DIP\_b to DIP\_a & affected
by temperature, viscosity of water, concentration gradient, distance,
and tortuosity of fluid matrix (Wang et al.~2003; Reddy \& Delaune
2008)\tabularnewline
r\_sed\_IM & 1/d & proportional rate of sedimentation of IM\_a to IM\_b
& affected by settling velocity (temperature, viscosity of water,
particle radius, particle density) and water depth, set equal to 1 when
depth is less than settling velocity (Reddy \& Delaune
2008)\tabularnewline
r\_sed\_OM & 1/d & proportional rate of sedimentation of OM\_a to OM\_b
& same as r\_sed\_OM\tabularnewline
\bottomrule
\end{longtable}

\hypertarget{forcings-hydroclimatic-inputs}{%
\subsubsection{Forcings (Hydroclimatic
Inputs)}\label{forcings-hydroclimatic-inputs}}

The table below gives the variable names and assumptions for the forcing
variables used in the model. The model was forced with water level data
collected in situ and meteorological data from Burlington Int'l Airport
(NOAA NCDC). Water level was measured at field sites by HOBO MX2001
pressure and temperature sensors placed just below the soil surface.
Data was corrected for variation in local barometric pressure, also
measured by HOBO MX2001s. Any gaps in the water level sensor record were
filled via time lag regression with other sensors in the area or with
USGS guages (USGS NWIS, r\^{}2\textgreater0.9). Water temperature was
modeled from air temperature based using a statistical fit to with
miniDOT sensors at the soil water interface (r\^{}2\textgreater0.9).
Precipitation was taken as the daily totals from meteorological data.
Evapotranspiration rate was estimated using the penman monteith method
via the R package \texttt{evapotranspiration}, substituting sunshine
hours for solar radiation. Water volume were calculated from area,
porosity (assumed = 1), and water depth. We caclulated the first
derivative in the time of water volume, and used this to solve for net
surface flow. Surface inflow and outflow were deduced from net surface
flow by adjusting for through flow. Through flow was calculated as the
volume of water divided by the days hydraulic residence time (HRT or
\(\tau\)).

\$\$

\text{HYDROLOGY SUBROUTINE:}\textbackslash{} V\_w =
A\rho\emph{aH\_w\textbackslash{} dV}\{w\} = V\_\{w,t\} - V\_\{w,t+1\}
\textbackslash{} Q\_\{net\}= dV\_w - A(ip - ET) \textbackslash{}
\text{IF Q}\emph{\text{net} \textgreater{} 0: \textbackslash{} Q}\{in\}
= Vw/\tau + Qnet \textbackslash{} Q\_\{in\} = -
V\_w/\tau \textbackslash{} \text{ELSE:} \textbackslash{} Q\_\{in\} =
V\_w/\tau \textbackslash{} Q\_\{out\} = V\_w/\tau - Q\_\{net\}
\textbackslash{}

\$\$

\begin{verbatim}
## Rows: 14 Columns: 4
\end{verbatim}

\begin{verbatim}
## -- Column specification --------------------------------------------------------
## Delimiter: "|"
## chr (4): Symbol ,  Units ,  Definition ,  Assumptions and Sources
\end{verbatim}

\begin{verbatim}
## 
## i Use `spec()` to retrieve the full column specification for this data.
## i Specify the column types or set `show_col_types = FALSE` to quiet this message.
\end{verbatim}

\begin{verbatim}
## Warning: One or more parsing issues, see `problems()` for details
\end{verbatim}

\begin{longtable}[]{@{}lccc@{}}
\caption{Table 1. Hydroclimate variables}\tabularnewline
\toprule
Symbol & Units & Definition & Assumptions and Sources\tabularnewline
\midrule
\endfirsthead
\toprule
Symbol & Units & Definition & Assumptions and Sources\tabularnewline
\midrule
\endhead
Zs & (m, NAD'83) & elevation of sediment surface & estimated from LiDAR
0.5m DEM (VCGI), corrected with Emlid Reach RS+ RTK/GNSS survey
(centimeter level accuracy)\tabularnewline
Hw & (m) & height of water above sediment surface & measured with HOBO
MX2001 water level logger\tabularnewline
Zw & (m, NAD'83) & elevation of water & Hw + Zs\tabularnewline
A & (m\^{}2) & wetland surface area & interpolated from stage table as
f(Hw)\tabularnewline
Vw & (m\^{}3) & Water volume of wetland surface water & calculated from
A and H\_w\tabularnewline
ET & (cm/day) & Evapotranspiration rate & Calculated at daily intervals
with penman monteith equation via the \texttt{Evapotranspiration}
package, weather data from BURLINGTON INTERNATIONAL AIRPORT, VT US
(WBAN:14742) (NOAA NCDC)\tabularnewline
ip & (cm/day) & Precipitation rate & totals derived from sub-hourly
weather observations from BURLINGTON INTERNATIONAL AIRPORT, VT US
(WBAN:14742)(NOAA NCDC)\tabularnewline
Qnet & (m\^{}3) & net surface flow & deduced from dVw, and A(ip -
ET)\tabularnewline
Qin & (m\^{}3/day) & Volumetric inflow rate & modeled with HydroCAD
and/or solved from water balance\tabularnewline
Qout & (m\^{}3/day) & Wetland discharge (outflow) rate & Modeled as a
f(Hw) based on site observations\tabularnewline
Qg & (m\^{}3/day) & Groundwater discharge (negative for infiltration) &
assumed = 0\tabularnewline
Uw & (m/s) & Wind speed & mean derived from sub-hourly data from
BURLINGTON INTERNATIONAL AIRPORT, VT US (WBAN:14742) \textbar{} used in
evapotranspiration calculation\tabularnewline
Tair & (\textless U+00B0\textgreater C) & Daily air temperature & mean
derived from sub-hourly data BURLINGTON INTERNATIONAL AIRPORT, VT US
(WBAN:14742) \textbar{}\tabularnewline
TW & (\textless U+00B0\textgreater C) & Daily water average temperature
& Modeled from Tair using equation from linear model fit to temperature
measured with PME miniDOT. IF(Tair \textgreater{} 0): TW = 2.5+0.8Tair
ELSE: TW = 0\tabularnewline
\bottomrule
\end{longtable}

\hypertarget{parameters}{%
\subsubsection{Parameters}\label{parameters}}

\begin{longtable}[]{@{}lllll@{}}
\caption{local (measured) parameters}\tabularnewline
\toprule
Name & Value & Unit & Description & Assumptions\tabularnewline
\midrule
\endfirsthead
\toprule
Name & Value & Unit & Description & Assumptions\tabularnewline
\midrule
\endhead
area & 1 & m\^{}2 & wetland surface area & uniform flat
surface\tabularnewline
H\_b & 0.1 & m & height of belowground compartment (sediment column) &
NA\tabularnewline
k\_HRT & 1e3 & d & hydraulic residence time of wetland surface water &
calculated by dividing total system water volume (m3) by outlfow rate
(m3/d), often changes as function of system water volume\tabularnewline
k\_TSS & 15 & g/m\^{}3 & total suspended solids of inflow & based on
field data, median of observations, 3.5 at prindle, 23.8 at union st,
12.25 at swamp rd\tabularnewline
k\_TP & 0.05 & g P/m\^{}3 & TP concentration (mg P /L) in inflow & 0.071
at prindle, 0.059 at union, 0.056 at swamp\tabularnewline
k\_LOI & 0.20 & g/g & initial fraction of organic matter in total mass
of below ground compartment & measured as soil
loss-on-ignition\tabularnewline
k\_PSR & 0.20 & mol/mol & P Saturation Ratio & molar ratio of oxalate
extractable P/(Al + Fe) (Nair et al.~2004), fit to field data, prindle
0.09 - 0.15, union 0.08 - 0.13, swamp rd 0.11 - 0.26\tabularnewline
k\_Ex\_max & 4 & g/kg & maximum P storage capacity & 31*(Al/27 + Fe/56),
where Al and Fe are determined by acid ammonium oxalate extraction, fit
to field data, ranging from 3.3 - 5.5 prindle, 5.0 - 6.4 union, 3.44 -
5.1 swamp\tabularnewline
k\_clay & 0.1 & g/g & clay content of inorganic matter, used for
particle settling velocity & from soil textural analysis OR from NRCS
soil survey units texture class, .11 to 0.35, .0875 to 0.15 union, 0.075
- .15 swamp\tabularnewline
k\_f\_fines & 0.90 & g/g & silt + clay, fine sediment fraction of
incoming total suspended solids used for particle setting velocity & fit
to field data and 0.627 - .84 prindle, 0.84 - 0.97 union, 0.75 - 0.985
swamp rd.\tabularnewline
k\_f\_OSS & 0.5 & g/g & organic matter fraction of incoming total
suspended solids & fit to field data, \%65 at prindle rd, 23\% at union
st, 54\% swamp rd.\tabularnewline
k\_f\_SRP & 0.3 & g SRP /g TP & fraction of TP as SRP in influent water
& based on field data 0.404 at prindle, 0.25 at union, 0.27 at swamp
rd.\tabularnewline
k\_DIP\_E & 0.05 & & equilibrium DIP concentration & used if
IO\_variable\_DIP\_E = F, set equal to final intact SRP for aerobic
treatments\tabularnewline
k\_rp\_i & fn\_particle\_radius(sand & & average radius of inorganic
particles & calculated based on soil texture see
\texttt{fn\_particle\_radius}\tabularnewline
\bottomrule
\end{longtable}

\begin{longtable}[]{@{}lllll@{}}
\caption{stochastic (unmeasured/calibrated) parameters}\tabularnewline
\toprule
Name & Value & Unit & Description & Assumptions\tabularnewline
\midrule
\endfirsthead
\toprule
Name & Value & Unit & Description & Assumptions\tabularnewline
\midrule
\endhead
k\_T\_STD & 13.75 & deg C & standard temperature for metabolic processes
& calibrated to make actual NPP match ANPPmax, since experiments were
conducted under field conditions this parameter is equal to the (maximum
daily average temp - minimum daily average temp)/2 + minimum daily
average temp \textasciitilde{} 15 - 17 degrees\tabularnewline
k\_SRP2PIP & 0.98 & g P/d.w. & ratio of LOP to SRP & 8.9e-1 for prindle,
1.42 for swamp rd, 6.2e-1 for union st\tabularnewline
k\_ISS2P & 0.0013 & g P/d.w. & P content of inorganic suspended
sediments & site data 0.002 for prindle rd, 0.0009 for union st, 0.00094
for swamp rd\tabularnewline
k\_shootM & 0 & g dw/m2 & shoot live biomass & need to set up a way to
get this to vary based on start time\tabularnewline
k\_rootM & 1000 & g dw/m2 & shoot live biomass & need to set up a way to
get this to vary based on start time\tabularnewline
k\_BM2P & 0.001 & g/g P/d.w. & P content of biomass & McJannet et
al.~1996 .001 - 0.003; Morris \& Bowden 1986 0.002; Wiegman Ch 2 data
0.001 to 0.003\tabularnewline
k\_f\_G\_shoot & 0.5 & fraction & fraction of NPP allocated to shoot
growth (shoot\_NPP/total\_NPP) & Morris et al 1984 0.2 -
0.5\tabularnewline
k\_NPP & 1500 & g m-2 y-1 & combined annual rate of NPP for above and
below ground biomass & Morris et al.~1984 1000 to 4000\tabularnewline
k\_ADNPP & k\_NPP/365 & g m-2 d-1 & average daily rate of NPP & divide
k\_NPP by 365\tabularnewline
k\_M & 0.001 & 1/day & rate of baseline biomass mortality & calibrated
to root mass \textasciitilde1000 - 2000 g m-2 and peak shootM
\textasciitilde300-800 g m-2 at use 0.003 for k\_ANPPmax = 3000, with
guidance from Morris et al 1984 0.003 to 0.007; Marois \& Mitsch 2016
0.0005 - 0.007\tabularnewline
k\_M\_shoot\_T\_mult & 50 & factor & multiplier for shoot mortality
after temp drops below threshold & calibrated to field
observations\tabularnewline
k\_T\_thresh\_M\_shoot & 6 & deg C & temperature at which shoot
mortality increases & calibrated to field observations\tabularnewline
k\_whc & 1e-3 & & a small volume of water to prevent errors associated
with empty compartments & best guess based on fit of oven dry verses air
dry moisture content\tabularnewline
k\_diff\_STD & 1e-1 & m\^{}2/d & effective diffusion coefficient &
calibrated to intact core data; Marois \& Mitsch 2016 calibrated value
was 2e-5 m2 d-1\tabularnewline
k\_ad & 1.75 & 1/d & adsorption first order rate coefficient & Wang et
al.~2003 1.75, Marois \& Mitsch 2016 used\tabularnewline
k\_E & .56 & m\^{}3/g & langmuir constant of adsorption (bond energy) &
Calibrated to intact core data this value depends on what metric is used
to define Ex\_max, Wang et al.~2003 2.75 m3 kg-1\tabularnewline
k\_PIP2Ex & 1 & g/g & ratio of exchangeable P to particulate inorganic P
& Wang et al.~2003 0.8\tabularnewline
k\_f\_labile & 0.8 & g/g & labile fraction organic matter & Morris \&
Bowden 1986 refractory fraction of 0.2 k\_f\_LOM\_OSS = k\_f\_labile \#
g/g\tabularnewline
k\_decay\_litter & 0.01 & 1/d & litter decomposition rate coefficient at
STD temp & Morris \& Bowden, Wiegman Ch 3, \# Longhi et al.~2008 k =
ranged from 0.01 1/d to 0.0027 1/d\tabularnewline
k\_decay\_LOP & 0.01 & 1/d & LOP decomposition rate coefficient at STD
temp & Marois \& Mitsch 2016 DOP rate is 0.01, while LPOP rate is 0.003,
since we do not model DOP LOP decay should be between 0.001 -
0.01\tabularnewline
k\_decay\_ROP & 1e-5 & 1/d & ROP decomposition rate coefficient at STD
temp when soils are unsaturated and aerobic(H\_w \textless{} 0) & Morris
\& Bowden 1986 assume refractory OM does not decompose, however this is
assuming saturated soils, so we assume that when H\_w \textless{} 0 that
ROP decomposes at between 1e-5 and 5e-5 based on value from Marois \&
Mitsch 2016 of 2.5e-5\tabularnewline
k\_rp\_o & 4.5e-7 & m & average radius of organic particles & Marois \&
Mitsch 2016\tabularnewline
\bottomrule
\end{longtable}

\begin{longtable}[]{@{}lllll@{}}
\caption{universal constant parameters}\tabularnewline
\toprule
Name & Value & Unit & Description & Assumptions\tabularnewline
\midrule
\endfirsthead
\toprule
Name & Value & Unit & Description & Assumptions\tabularnewline
\midrule
\endhead
k\_dp\_i & 2.65e6 & g/m\^{}3 & particle density of inorganic matter &
Delaune et al.~1983 g/cm\^{}3 * 10\^{}6
cm\textsuperscript{3/m}3\tabularnewline
k\_dp\_o & 1.14e6 & g/m\^{}3 & particle density of inorganic matter &
Delaune et al.~1983 g/cm\^{}3 * 10\^{}6
cm\textsuperscript{3/m}3\tabularnewline
k\_db\_i & 1.99e6 & g/m\^{}3 & bulk density of inorganic matter & Morris
et al.~2016\tabularnewline
k\_db\_o & 0.085e6 & g/m\^{}3 & bulk density of organic matter & Morris
et al.~2016\tabularnewline
k\_pi & 3.141593 & & arc length of a circle &\tabularnewline
k\_g & 7.32e10 & m/d\^{}2 & acceleration due to gravity &
constant\tabularnewline
k\_mew & 86.4e3 & g/m/d & viscosity of water & standard
value\tabularnewline
k\_dw & 1e6 & g/m\^{}3 & density of water & 1e6 for 0 salinity, 1.025e6
for 34 ppm salinity water\tabularnewline
k\_diff & 1.931741e-05 & m\^{}2/d & effective diffusion coefficient &
standard tempurature and pressure see fn\_kDiff\tabularnewline
\bottomrule
\end{longtable}

\begin{longtable}[]{@{}lllll@{}}
\caption{run specifications parameters}\tabularnewline
\toprule
Name & Value & Unit & Description & Assumptions\tabularnewline
\midrule
\endfirsthead
\toprule
Name & Value & Unit & Description & Assumptions\tabularnewline
\midrule
\endhead
version & ``wetlandPv02'' & chr & name of the model version
&\tabularnewline
simname & ``default'' & chr & name of the model simulation
&\tabularnewline
simtype & ``static'' & chr & charater string indicating the objective of
the model run used & ``static'' for steady sate, ``forecast'' for
projections and scenario analysis, and ``calibration'' for
training/calibration\tabularnewline
startday & 0 & d & julian day (0-365) of simulation start & based period
of forcing data\tabularnewline
simyears & 14/365 & y & number of years in simulation &
""\tabularnewline
increment & 1 & d & number of days in each time step of model & if not
equal to 1 then accuracy of simulation needs to be
verified\tabularnewline
extended\_outputs & T & logical & True/False indicating if the purpose
of the run is to debug & if so writes extended outputs, this
significantly slows the run time\tabularnewline
IO\_Q\_in & T & logical & toggles surface inflow & T/F or
1/0\tabularnewline
IO\_Q\_precip & T & logical & toggle precipitation &\tabularnewline
IO\_Q\_ground & T & logical & toggles net groundwater flow (percolation
- infiltration) &\tabularnewline
IO\_Q\_ET & T & logical & toggles evapotranspiration &\tabularnewline
IO\_Q\_out & T & logical & toggles surface outflow &\tabularnewline
IO\_assim\_shootP & T & logical & toggles assimilation of shoot P
&\tabularnewline
IO\_assim\_rootP & T & logical & toggles growth of root P
&\tabularnewline
IO\_mort\_shootP2litterP & T & logical & toggles mortality of shoots
&\tabularnewline
IO\_mort\_rootP2LOP & T & logical & toggles mortality of root P to LOP
&\tabularnewline
IO\_mort\_rootP2ROP & T & logical & toggles mortatlity of root P
&\tabularnewline
IO\_sed\_IM & T & logical & toggles sedimentation of inorganic matter
&\tabularnewline
IO\_decay\_litter & T & logical & toggles decomposition of litter P to
refractory organic P &\tabularnewline
IO\_decay\_LOP & T & logical & toggles decomposition of labile OP
&\tabularnewline
IO\_decay\_ROP & T & logical & toggles decomposition of refractory OP
&\tabularnewline
IO\_diffus & T & logical & toggles diffusion of DIP from b to a
&\tabularnewline
IO\_adsorp & T & logical & toggles adsorption of DIP onto PIP
&\tabularnewline
IO\_in\_IM & T & logical & toggles inflow of inorganic matter as ISS
&\tabularnewline
IO\_in\_PIP & T & logical & toggles inflow of PIP &\tabularnewline
IO\_in\_LOP & T & logical & toggles inflow of labile organic P
&\tabularnewline
IO\_in\_ROP & T & logical & toggles inflow of recalcitrant organic P
&\tabularnewline
IO\_in\_DIP & T & logical & toggles inflow of dissolved inorganic P
&\tabularnewline
IO\_out\_IM & T & logical & toggles outflow of IM &\tabularnewline
IO\_out\_PIP & T & logical & toggles outflow of PIP &\tabularnewline
IO\_out\_LOP & T & logical & toggles outflow of LOP &\tabularnewline
IO\_out\_ROP & T & logical & toggles outflow of ROP &\tabularnewline
IO\_out\_DIP & T & logical & toggles outflow of DIP &\tabularnewline
IO\_DIP\_E\_langmuir & F & logical & turns on the use of langmuir model
for caclulating DIP\_E &\tabularnewline
IO\_variable\_k\_E & T & logical & toggles variable calculation of k\_E
&\tabularnewline
IO\_variable\_k\_Ex\_max & F & logical & toggles on variable calculation
of Ex\_max using statistical fit to fines and LOI &\tabularnewline
IO\_anoxic & F & logical & toggles anaerobic conditions for DIP\_E
concentration &\tabularnewline
IO\_variable\_DIP\_E & F & logical & toggles variable calculation of
DIP\_E & if = F, then k\_DIP\_E is used\tabularnewline
IO\_Q\_net & T & logical & toggles calculation of inflow and outflow
from Qnet, Vw and HRT & see hydrology subroutine\tabularnewline
IO\_HRT\_power\_model & F & logical & toggles calculation HRT from a
power model & if Zw \textgreater{} 0, HRT = a*Zw\^{}b, where Zw is
elevation relative to lowest elevation in the wetland, and
b\textless0\tabularnewline
\bottomrule
\end{longtable}

\hypertarget{differential-equations}{%
\subsubsection{Differential Equations}\label{differential-equations}}

Differential Equations for the model are generated from stoicheometry
matrix of the \textbf{state variables} and \textbf{process flows} (see
\textbf{``mass balance''}).

\begin{longtable}[]{@{}ll@{}}
\caption{Differential equations for model states}\tabularnewline
\toprule
Name & Value\tabularnewline
\midrule
\endfirsthead
\toprule
Name & Value\tabularnewline
\midrule
\endhead
d\_IM\_a & -1 * sed\_IM + 1 * in\_IM + -1 * out\_IM\tabularnewline
d\_IM\_b & 1 * sed\_IM\tabularnewline
d\_shootP & 1 * assim\_shootP + -1 * mort\_shootP2litterP\tabularnewline
d\_rootP & 1 * assim\_rootP + -1 * mort\_rootP2LOP + -1 *
mort\_rootP2ROP\tabularnewline
d\_litterP & 1 * mort\_shootP2litterP + -1 * dec\_litter2LOP\_a + -1 *
dec\_litter2ROP\_a\tabularnewline
d\_ROP\_a & -1 * sed\_ROP + 1 * dec\_litter2ROP\_a + -1 * dec\_ROP\_a +
1 * in\_ROP + -1 * out\_ROP\tabularnewline
d\_LOP\_a & -1 * sed\_LOP + 1 * dec\_litter2LOP\_a + -1 * dec\_LOP\_a +
1 * dec\_ROP\_a + 1 * in\_LOP + -1 * out\_LOP\tabularnewline
d\_PIP\_a & -1 * sed\_PIP + 1 * in\_PIP + -1 * out\_PIP\tabularnewline
d\_DIP\_a & 1 * dec\_LOP\_a + 1 * diff\_DIP\_b2a + 1 * in\_DIP + -1 *
out\_DIP\tabularnewline
d\_ROP\_b & 1 * mort\_rootP2ROP + 1 * sed\_ROP + -1 *
dec\_ROP\_b\tabularnewline
d\_LOP\_b & 1 * mort\_rootP2LOP + 1 * sed\_LOP + -1 * dec\_LOP\_b + 1 *
dec\_ROP\_b\tabularnewline
d\_PIP\_b & 1 * sed\_PIP + 1 * sorp\_DIP2PIP\_b\tabularnewline
d\_DIP\_b & -1 * assim\_shootP + -1 * assim\_rootP + 1 * dec\_LOP\_b +
-1 * diff\_DIP\_b2a + -1 * sorp\_DIP2PIP\_b\tabularnewline
\bottomrule
\end{longtable}

\hypertarget{mass-balance}{%
\subsubsection{Mass balance}\label{mass-balance}}

Differential Equations for the model are generated from stoicheometry
matrix of the \textbf{state variables} (state or states for short) and
\textbf{process flows} (see \texttt{stoicheometry.xlsx}). In this matrix
the modeler enters a value of 1 (adding to a state), -1 (subtracting
from state) or blank (not interacting with a state) for each combination
of a state variable and a process flow. The table below contains the
stoicheometry matrix for the current model. Note the column
\texttt{balance} is the row sum for a given process, values above or
below than zero indicates that a process adds/removes mass from the
model domain, while a balance of zero indicates that a process is
conservative (does not affect the total mass in the domain).

\begin{longtable}[]{@{}rlrrrrrrrrrrrrr@{}}
\caption{parameters (numeric constants and run
specifications)}\tabularnewline
\toprule
balance & variables (right) processes (below) & IM\_a & IM\_b & shootP &
rootP & litterP & ROP\_a & LOP\_a & PIP\_a & DIP\_a & ROP\_b & LOP\_b &
PIP\_b & DIP\_b\tabularnewline
\midrule
\endfirsthead
\toprule
balance & variables (right) processes (below) & IM\_a & IM\_b & shootP &
rootP & litterP & ROP\_a & LOP\_a & PIP\_a & DIP\_a & ROP\_b & LOP\_b &
PIP\_b & DIP\_b\tabularnewline
\midrule
\endhead
0 & assim\_shootP & & & 1 & & & & & & & & & & -1\tabularnewline
0 & assim\_rootP & & & & 1 & & & & & & & & & -1\tabularnewline
0 & mort\_shootP2litterP & & & -1 & & 1 & & & & & & & &\tabularnewline
0 & mort\_rootP2LOP & & & & -1 & & & & & & & 1 & &\tabularnewline
0 & mort\_rootP2ROP & & & & -1 & & & & & & 1 & & &\tabularnewline
0 & sed\_IM & -1 & 1 & & & & & & & & & & &\tabularnewline
0 & sed\_PIP & & & & & & & & -1 & & & & 1 &\tabularnewline
0 & sed\_LOP & & & & & & & -1 & & & & 1 & &\tabularnewline
0 & sed\_ROP & & & & & & -1 & & & & 1 & & &\tabularnewline
0 & dec\_litter2LOP\_a & & & & & -1 & & 1 & & & & & &\tabularnewline
0 & dec\_litter2ROP\_a & & & & & -1 & 1 & & & & & & &\tabularnewline
0 & dec\_LOP\_a & & & & & & & -1 & & 1 & & & &\tabularnewline
0 & dec\_ROP\_a & & & & & & -1 & 1 & & & & & &\tabularnewline
0 & dec\_LOP\_b & & & & & & & & & & & -1 & & 1\tabularnewline
0 & dec\_ROP\_b & & & & & & & & & & -1 & 1 & &\tabularnewline
0 & diff\_DIP\_b2a & & & & & & & & & 1 & & & & -1\tabularnewline
0 & sorp\_DIP2PIP\_b & & & & & & & & & & & & 1 & -1\tabularnewline
1 & in\_IM & 1 & & & & & & & & & & & &\tabularnewline
1 & in\_PIP & & & & & & & & 1 & & & & &\tabularnewline
1 & in\_LOP & & & & & & & 1 & & & & & &\tabularnewline
1 & in\_ROP & & & & & & 1 & & & & & & &\tabularnewline
1 & in\_DIP & & & & & & & & & 1 & & & &\tabularnewline
-1 & out\_IM & -1 & & & & & & & & & & & &\tabularnewline
-1 & out\_PIP & & & & & & & & -1 & & & & &\tabularnewline
-1 & out\_LOP & & & & & & & -1 & & & & & &\tabularnewline
-1 & out\_ROP & & & & & & -1 & & & & & & &\tabularnewline
-1 & out\_DIP & & & & & & & & & -1 & & & &\tabularnewline
\bottomrule
\end{longtable}

\hypertarget{numerical-stability-checks}{%
\subsubsection{Numerical Stability
Checks}\label{numerical-stability-checks}}

The following figures verify the performance of the model under
increasing complexity of simulation.

\hypertarget{all-stocks-should-be-constant-through-time}{%
\paragraph{1. all stocks should be constant through
time}\label{all-stocks-should-be-constant-through-time}}

\begin{figure}
\centering
\includegraphics{C:/Workspace/wetlandP/model_versions/wetlandP_v2/wetlandP_v2.1/documentation//fig1_states_W0_B0_G0.png}
\caption{1. all stocks should be constant through time}
\end{figure}

\hypertarget{dip-and-pip-should-equilibrate-no-other-stocks-should-change}{%
\paragraph{2. DIP and PIP should equilibrate, no other stocks should
change}\label{dip-and-pip-should-equilibrate-no-other-stocks-should-change}}

\begin{figure}
\centering
\includegraphics{C:/Workspace/wetlandP/model_versions/wetlandP_v2/wetlandP_v2.1/documentation//fig2_states_W0_B0_G1.png}
\caption{2. DIP and PIP should equilibrate, no other stocks should
change}
\end{figure}

\hypertarget{shootp-rootp-lop-rop-should-fluctuuate-dip-and-pip-should-be-constant}{%
\paragraph{3. shootP, rootP, LOP, ROP should fluctuuate, DIP and PIP
should be
constant}\label{shootp-rootp-lop-rop-should-fluctuuate-dip-and-pip-should-be-constant}}

\begin{figure}
\centering
\includegraphics{C:/Workspace/wetlandP/model_versions/wetlandP_v2/wetlandP_v2.1/documentation//fig3_states_W0_B1_G0.png}
\caption{3. shootP, rootP, LOP, ROP should fluctuuate, DIP and PIP
should be constant}
\end{figure}

\hypertarget{all-state-variables-should-fluctuate}{%
\paragraph{4. all state variables should
fluctuate}\label{all-state-variables-should-fluctuate}}

\begin{figure}
\centering
\includegraphics{C:/Workspace/wetlandP/model_versions/wetlandP_v2/wetlandP_v2.1/documentation//fig4_states_W0_B1_G1.png}
\caption{4. all state variables should fluctuate}
\end{figure}

\hypertarget{volume-of-water-should-be-constant-through-time}{%
\paragraph{5. volume of water should be constant through
time}\label{volume-of-water-should-be-constant-through-time}}

\begin{figure}
\centering
\includegraphics{C:/Workspace/wetlandP/model_versions/wetlandP_v2/wetlandP_v2.1/documentation//fig5_hydroclimate_static_W1_B0_G0.png}
\caption{5. volume of water should be constant through time}
\end{figure}

\hypertarget{hydrocliamte-data-being-forced-on-the-model}{%
\paragraph{6. hydrocliamte data being forced on the
model}\label{hydrocliamte-data-being-forced-on-the-model}}

\begin{figure}
\centering
\includegraphics{C:/Workspace/wetlandP/model_versions/wetlandP_v2/wetlandP_v2.1/documentation//fig6_hydroclimate_W1_B0_G0.png}
\caption{6. hydrocliamte data being forced on the model}
\end{figure}

\hypertarget{all-states-should-fluctuate-but-there-should-be-no-discontinuities-or-negative-values-inorganic-matter-compartment-should-be-constant-since-tss-0}{%
\paragraph{7. all states should fluctuate but there should be no
discontinuities, or negative values, inorganic matter compartment should
be constant since TSS =
0}\label{all-states-should-fluctuate-but-there-should-be-no-discontinuities-or-negative-values-inorganic-matter-compartment-should-be-constant-since-tss-0}}

\begin{figure}
\centering
\includegraphics{C:/Workspace/wetlandP/model_versions/wetlandP_v2/wetlandP_v2.1/documentation//fig7_states_W1_B1_G1.png}
\caption{7. all states should fluctuate but there should be no
discontinuities, or negative values, inorganic matter compartment should
be constant since TSS = 0}
\end{figure}

\hypertarget{concentrations-shoudl-fluctuate-but-have-no-sharp-discontinuities-or-negative-values}{%
\paragraph{8. concentrations shoudl fluctuate but have no sharp
discontinuities, or negative
values}\label{concentrations-shoudl-fluctuate-but-have-no-sharp-discontinuities-or-negative-values}}

\begin{figure}
\centering
\includegraphics{C:/Workspace/wetlandP/model_versions/wetlandP_v2/wetlandP_v2.1/documentation//fig8_DIP_A_W1_B1_G1.png}
\caption{8. concentrations shoudl fluctuate but have no sharp
discontinuities, or negative values}
\end{figure}

\hypertarget{references}{%
\subsection{References}\label{references}}

\hypertarget{online-data-sources}{%
\subsubsection{Online Data Sources}\label{online-data-sources}}

NOAA NCDC. National Oceanographic and Atmospheric Administration,
National Centers for Environmental Information, National Climatic Data
Center. United States Department of Commerce. URL: www.noaa.ncdc.gov
(accessed on 2021-10-25).

USGS NWIS. United States Geologic Survey, National Water Information
System. United States Department of the Interior. URL:
www.waterdata.usgs.gov (accessed on 2021-10-25).

VCGI. Vermont Open Geodata Portal, Vermont Center for Geographic
Information. AGENCY OF DIGITAL SERVICES. URL: www.geodata.vermont.gov
(acessed on 2021-10-25)

\hypertarget{scientific-literature}{%
\subsubsection{Scientific Literature}\label{scientific-literature}}

DeLaune, R. D., Baumann, R. H., \& Gosselink, J. G. (1983).
Relationships among Vertical Accretion, Coastal Submergence, and Erosion
in a Louisiana Gulf Coast Marsh. SEPM Journal of Sedimentary Research,
53(1), 147--157.
\url{https://doi.org/10.1306/212F8175-2B24-11D7-8648000102C1865D}

Hantush, M. M., Kalin, L., Isik, S., \& Yucekaya, A. (2013). Nutrient
Dynamics in Flooded Wetlands. I: Model Development. Journal of
Hydrologic Engineering, 18(12), 1709--1723.
\url{https://doi.org/10.1061/(ASCE)HE.1943-5584.0000741}

Marois, D. E., \& Mitsch, W. J. (2016). Modeling phosphorus retention at
low concentrations in Florida Everglades mesocosms. Ecological
Modelling, 319, 42-62.

Morris, J. T., Houghton, R. A., \& Botkin, D. B. (1984). Theoretical
limits of belowground production by Spartina alterniflora: An analysis
through modelling. Ecological Modelling, 26(3--4), 155--175.
\url{https://doi.org/10.1016/0304-3800(84)90068-1}

Morris, J. T., \& Bowden, W. B. (1986). A Mechanistic, Numerical Model
of Sedimentation, Mineralization, and Decomposition for Marsh
Sediments1. Soil Science Society of America Journal, 50(1), 96.
\url{https://doi.org/10.2136/sssaj1986.03615995005000010019x}

Morris, J. T., Sundareshwar, P. V., Nietch, C. T., Kjerfve, B., \&
Cahoon, D. R. (2002). Responses of coastal wetlands to rising sea level.
Ecology, 83(10), 2869-2877.

Morris, J. T., Barber, D. C., Callaway, J. C., Chambers, R., Hagen, S.
C., Hopkinson, C. S., \ldots{} Wigand, C. (2016). Contributions of
organic and inorganic matter to sediment volume and accretion in tidal
wetlands at steady state. Earth's Future, 4(4), 110--121.
\url{https://doi.org/de}

Reddy, K. R., \& Delaune, R. D. (2008). Biochemistry of wetland science
and application. CRC Press Taylor \& Francis Group, Boca Raton FL. ISBN
978-1-56670-678-0

Wang, N., \& Mitsch, W. J. (2000). A detailed ecosystem model of
phosphorus dynamics in created riparian wetlands. Ecological Modelling,
126(2--3), 101--130. \url{https://doi.org/10.1016/S0304-3800(00)00260-X}

Wang, H., Appan, A., \& Gulliver, J. S. (2003). Modeling of phosphorus
dynamics in aquatic sediments: I - Model development. Water Research.
\url{https://doi.org/10.1016/S0043-1354(03)00304-X}

Wiegman, A. R. H., Day, J. W., D'Elia, C. F., Rutherford, J. S., Morris,
J. T., Roy, E. D., \ldots{} Snyder, B. F. (2018). Modeling impacts of
sea-level rise, oil price, and management strategy on the costs of
sustaining Mississippi delta marshes with hydraulic dredging. Science of
the Total Environment, 618, 1547--1559.
\url{https://doi.org/10.1016/j.scitotenv.2017.09.314}

\hypertarget{software-dependancies}{%
\subsubsection{Software Dependancies}\label{software-dependancies}}

\$R

To cite R in publications use:

R Core Team (2020). R: A language and environment for statistical
computing. R Foundation for Statistical Computing, Vienna, Austria. URL
\url{https://www.R-project.org/}.

A BibTeX entry for LaTeX users is

@Manual\{, title = \{R: A Language and Environment for Statistical
Computing\}, author = \{\{R Core Team\}\}, organization = \{R Foundation
for Statistical Computing\}, address = \{Vienna, Austria\}, year =
\{2020\}, url = \{\url{https://www.R-project.org/}\}, \}

We have invested a lot of time and effort in creating R, please cite it
when using it for data analysis. See also `citation(``pkgname'')' for
citing R packages.

\$ggrepel

To cite package `ggrepel' in publications use:

Kamil Slowikowski (2021). ggrepel: Automatically Position
Non-Overlapping Text Labels with `ggplot2'. R package version 0.9.1.
\url{https://CRAN.R-project.org/package=ggrepel}

A BibTeX entry for LaTeX users is

@Manual\{, title = \{ggrepel: Automatically Position Non-Overlapping
Text Labels with `ggplot2'\}, author = \{Kamil Slowikowski\}, year =
\{2021\}, note = \{R package version 0.9.1\}, url =
\{\url{https://CRAN.R-project.org/package=ggrepel}\}, \}

\$ecolMod

To cite package `ecolMod' in publications use:

Karline Soetaert and Peter MJ Herman (2014). ecolMod: ``A practical
guide to ecological modelling - using R as a simulation platform''. R
package version 1.2.6. \url{https://CRAN.R-project.org/package=ecolMod}

A BibTeX entry for LaTeX users is

@Manual\{, title = \{ecolMod: ``A practical guide to ecological
modelling - using R as a simulation platform''\}, author = \{Karline
Soetaert and Peter MJ Herman\}, year = \{2014\}, note = \{R package
version 1.2.6\}, url =
\{\url{https://CRAN.R-project.org/package=ecolMod}\}, \}

ATTENTION: This citation information has been auto-generated from the
package DESCRIPTION file and may need manual editing, see
`help(``citation'')'.

\$diagram

To cite package `diagram' in publications use:

Karline Soetaert (2020). diagram: Functions for Visualising Simple
Graphs (Networks), Plotting Flow Diagrams. R package version 1.6.5.
\url{https://CRAN.R-project.org/package=diagram}

A BibTeX entry for LaTeX users is

@Manual\{, title = \{diagram: Functions for Visualising Simple Graphs
(Networks), Plotting Flow Diagrams\}, author = \{Karline Soetaert\},
year = \{2020\}, note = \{R package version 1.6.5\}, url =
\{\url{https://CRAN.R-project.org/package=diagram}\}, \}

ATTENTION: This citation information has been auto-generated from the
package DESCRIPTION file and may need manual editing, see
`help(``citation'')'.

\$shape

To cite package `shape' in publications use:

Karline Soetaert (2021). shape: Functions for Plotting Graphical Shapes,
Colors. R package version 1.4.6.
\url{https://CRAN.R-project.org/package=shape}

A BibTeX entry for LaTeX users is

@Manual\{, title = \{shape: Functions for Plotting Graphical Shapes,
Colors\}, author = \{Karline Soetaert\}, year = \{2021\}, note = \{R
package version 1.4.6\}, url =
\{\url{https://CRAN.R-project.org/package=shape}\}, \}

ATTENTION: This citation information has been auto-generated from the
package DESCRIPTION file and may need manual editing, see
`help(``citation'')'.

\$rootSolve

To cite package `rootSolve' in publications use:

Soetaert K. and P.M.J. Herman (2009). A Practical Guide to Ecological
Modelling. Using R as a Simulation Platform. Springer, 372 pp.

Soetaert K. (2009). rootSolve: Nonlinear root finding, equilibrium and
steady-state analysis of ordinary differential equations. R-package
version 1.6

rootSolve was created to solve the examples from chapter 7 of our book -
please cite this book when using it, thank you! To see these entries in
BibTeX format, use `print(, bibtex=TRUE)', `toBibtex(.)', or set
`options(citation.bibtex.max=999)'.

\$rlang

To cite package `rlang' in publications use:

Lionel Henry and Hadley Wickham (2021). rlang: Functions for Base Types
and Core R and `Tidyverse' Features. R package version 0.4.11.
\url{https://CRAN.R-project.org/package=rlang}

A BibTeX entry for LaTeX users is

@Manual\{, title = \{rlang: Functions for Base Types and Core R and
`Tidyverse' Features\}, author = \{Lionel Henry and Hadley Wickham\},
year = \{2021\}, note = \{R package version 0.4.11\}, url =
\{\url{https://CRAN.R-project.org/package=rlang}\}, \}

\$Evapotranspiration

To cite package `Evapotranspiration' in publications use:

Danlu Guo, Seth Westra and Tim Peterson (2020). Evapotranspiration:
Modelling Actual, Potential and Reference Crop Evapotranspiration. R
package version 1.15.
\url{https://CRAN.R-project.org/package=Evapotranspiration}

A BibTeX entry for LaTeX users is

@Manual\{, title = \{Evapotranspiration: Modelling Actual, Potential and
Reference Crop Evapotranspiration\}, author = \{Danlu Guo and Seth
Westra and Tim Peterson\}, year = \{2020\}, note = \{R package version
1.15\}, url =
\{\url{https://CRAN.R-project.org/package=Evapotranspiration}\}, \}

ATTENTION: This citation information has been auto-generated from the
package DESCRIPTION file and may need manual editing, see
`help(``citation'')'.

\$soiltexture

To cite package `soiltexture' in publications use:

Julien Moeys (2018). soiltexture: Functions for Soil Texture Plot,
Classification and Transformation. R package version 1.5.1.
\url{https://CRAN.R-project.org/package=soiltexture}

A BibTeX entry for LaTeX users is

@Manual\{, title = \{soiltexture: Functions for Soil Texture Plot,
Classification and Transformation\}, author = \{Julien Moeys\}, year =
\{2018\}, note = \{R package version 1.5.1\}, url =
\{\url{https://CRAN.R-project.org/package=soiltexture}\}, \}

\$zoo

To cite zoo in publications use:

Achim Zeileis and Gabor Grothendieck (2005). zoo: S3 Infrastructure for
Regular and Irregular Time Series. Journal of Statistical Software,
14(6), 1-27. \url{doi:10.18637/jss.v014.i06}

A BibTeX entry for LaTeX users is

@Article\{, title = \{zoo: S3 Infrastructure for Regular and Irregular
Time Series\}, author = \{Achim Zeileis and Gabor Grothendieck\},
journal = \{Journal of Statistical Software\}, year = \{2005\}, volume =
\{14\}, number = \{6\}, pages = \{1--27\}, doi =
\{10.18637/jss.v014.i06\}, \}

\$diffeqr

Rackauckas C, Nie Q (2017). ``DifferentialEquations.jl -- A Performant
and Feature-Rich Ecosystem for Solving Differential Equations in
Julia.'' \emph{The Journal of Open Source Software}, \emph{5}(1). doi:
10.5334/jors.151 (URL: \url{https://doi.org/10.5334/jors.151}), R
package version 1.1.1, \textless URL:
\url{https://openresearchsoftware.metajnl.com/articles/10.5334/jors.151/}\textgreater.

A BibTeX entry for LaTeX users is

@Article\{, doi = \{10.5334/jors.151\}, journal = \{The Journal of Open
Source Software\}, title = \{DifferentialEquations.jl -- A Performant
and Feature-Rich Ecosystem for Solving Differential Equations in
Julia\}, author = \{Chris Rackauckas and Qing Nie\}, year = \{2017\},
volume = \{5\}, number = \{1\}, url =
\{\url{https://openresearchsoftware.metajnl.com/articles/10.5334/jors.151/}\},
note = \{R package version 1.1.1\}, \}

\$deSolve

To cite package `deSolve' in publications use:

Karline Soetaert, Thomas Petzoldt, R. Woodrow Setzer (2010). Solving
Differential Equations in R: Package deSolve. Journal of Statistical
Software, 33(9), 1--25. URL \url{http://www.jstatsoft.org/v33/i09/} DOI
10.18637/jss.v033.i09

A BibTeX entry for LaTeX users is

@Article\{, title = \{Solving Differential Equations in \{R\}: Package
de\{S\}olve\}, author = \{Karline Soetaert and Thomas Petzoldt and R.
Woodrow Setzer\}, journal = \{Journal of Statistical Software\}, volume
= \{33\}, number = \{9\}, pages = \{1--25\}, year = \{2010\}, coden =
\{JSSOBK\}, issn = \{1548-7660\}, url =
\{\url{http://www.jstatsoft.org/v33/i09}\}, doi =
\{10.18637/jss.v033.i09\}, keywords = \{ordinary differential equations,
partial differential equations, differential algebraic equations,
initial value problems, R, FORTRAN, C\}, \}

\$pacman

To cite pacman in publications, please use:

Rinker, T. W. \& Kurkiewicz, D. (2017). pacman: Package Management for
R. version 0.5.0. Buffalo, New York.
\url{http://github.com/trinker/pacman}

A BibTeX entry for LaTeX users is

@Manual\{, title = \{\{pacman\}: \{P\}ackage Management for \{R\}\},
author = \{Tyler W. Rinker and Dason Kurkiewicz\}, address = \{Buffalo,
New York\}, note = \{version 0.5.0\}, year = \{2018\}, url =
\{\url{http://github.com/trinker/pacman}\}, \}

\$forcats

To cite package `forcats' in publications use:

Hadley Wickham (2021). forcats: Tools for Working with Categorical
Variables (Factors). R package version 0.5.1.
\url{https://CRAN.R-project.org/package=forcats}

A BibTeX entry for LaTeX users is

@Manual\{, title = \{forcats: Tools for Working with Categorical
Variables (Factors)\}, author = \{Hadley Wickham\}, year = \{2021\},
note = \{R package version 0.5.1\}, url =
\{\url{https://CRAN.R-project.org/package=forcats}\}, \}

\$stringr

To cite package `stringr' in publications use:

Hadley Wickham (2019). stringr: Simple, Consistent Wrappers for Common
String Operations. R package version 1.4.0.
\url{https://CRAN.R-project.org/package=stringr}

A BibTeX entry for LaTeX users is

@Manual\{, title = \{stringr: Simple, Consistent Wrappers for Common
String Operations\}, author = \{Hadley Wickham\}, year = \{2019\}, note
= \{R package version 1.4.0\}, url =
\{\url{https://CRAN.R-project.org/package=stringr}\}, \}

\$dplyr

To cite package `dplyr' in publications use:

Hadley Wickham, Romain François, Lionel Henry and Kirill Müller (2021).
dplyr: A Grammar of Data Manipulation. R package version 1.0.7.
\url{https://CRAN.R-project.org/package=dplyr}

A BibTeX entry for LaTeX users is

@Manual\{, title = \{dplyr: A Grammar of Data Manipulation\}, author =
\{Hadley Wickham and Romain François and Lionel Henry and Kirill
Müller\}, year = \{2021\}, note = \{R package version 1.0.7\}, url =
\{\url{https://CRAN.R-project.org/package=dplyr}\}, \}

\$purrr

To cite package `purrr' in publications use:

Lionel Henry and Hadley Wickham (2020). purrr: Functional Programming
Tools. R package version 0.3.4.
\url{https://CRAN.R-project.org/package=purrr}

A BibTeX entry for LaTeX users is

@Manual\{, title = \{purrr: Functional Programming Tools\}, author =
\{Lionel Henry and Hadley Wickham\}, year = \{2020\}, note = \{R package
version 0.3.4\}, url =
\{\url{https://CRAN.R-project.org/package=purrr}\}, \}

\$readr

To cite package `readr' in publications use:

Hadley Wickham and Jim Hester (2021). readr: Read Rectangular Text Data.
R package version 2.0.1. \url{https://CRAN.R-project.org/package=readr}

A BibTeX entry for LaTeX users is

@Manual\{, title = \{readr: Read Rectangular Text Data\}, author =
\{Hadley Wickham and Jim Hester\}, year = \{2021\}, note = \{R package
version 2.0.1\}, url =
\{\url{https://CRAN.R-project.org/package=readr}\}, \}

\$tidyr

To cite package `tidyr' in publications use:

Hadley Wickham (2021). tidyr: Tidy Messy Data. R package version 1.1.3.
\url{https://CRAN.R-project.org/package=tidyr}

A BibTeX entry for LaTeX users is

@Manual\{, title = \{tidyr: Tidy Messy Data\}, author = \{Hadley
Wickham\}, year = \{2021\}, note = \{R package version 1.1.3\}, url =
\{\url{https://CRAN.R-project.org/package=tidyr}\}, \}

\$tibble

To cite package `tibble' in publications use:

Kirill Müller and Hadley Wickham (2021). tibble: Simple Data Frames. R
package version 3.1.4. \url{https://CRAN.R-project.org/package=tibble}

A BibTeX entry for LaTeX users is

@Manual\{, title = \{tibble: Simple Data Frames\}, author = \{Kirill
Müller and Hadley Wickham\}, year = \{2021\}, note = \{R package version
3.1.4\}, url = \{\url{https://CRAN.R-project.org/package=tibble}\}, \}

\$ggplot2

To cite ggplot2 in publications, please use:

H. Wickham. ggplot2: Elegant Graphics for Data Analysis. Springer-Verlag
New York, 2016.

A BibTeX entry for LaTeX users is

@Book\{, author = \{Hadley Wickham\}, title = \{ggplot2: Elegant
Graphics for Data Analysis\}, publisher = \{Springer-Verlag New York\},
year = \{2016\}, isbn = \{978-3-319-24277-4\}, url =
\{\url{https://ggplot2.tidyverse.org}\}, \}

\$tidyverse

Wickham et al., (2019). Welcome to the tidyverse. Journal of Open Source
Software, 4(43), 1686, \url{https://doi.org/10.21105/joss.01686}

A BibTeX entry for LaTeX users is

@Article\{, title = \{Welcome to the \{tidyverse\}\}, author = \{Hadley
Wickham and Mara Averick and Jennifer Bryan and Winston Chang and Lucy
D'Agostino McGowan and Romain François and Garrett Grolemund and Alex
Hayes and Lionel Henry and Jim Hester and Max Kuhn and Thomas Lin
Pedersen and Evan Miller and Stephan Milton Bache and Kirill Müller and
Jeroen Ooms and David Robinson and Dana Paige Seidel and Vitalie Spinu
and Kohske Takahashi and Davis Vaughan and Claus Wilke and Kara Woo and
Hiroaki Yutani\}, year = \{2019\}, journal = \{Journal of Open Source
Software\}, volume = \{4\}, number = \{43\}, pages = \{1686\}, doi =
\{10.21105/joss.01686\}, \}

\end{document}
